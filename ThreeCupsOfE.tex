\magnification 1200
\input /Users/jimwolper/TeX/standard.tex

\font\bigger=cmr10 scaled\magstep1
\font\smaller=cmr7
\def\Tau{\rm T}
%%%%%%%%%%%%%%%%%%%%%%%%%%%%%%%%%%%%%%%%%%%%%%%

\centerline{Idaho State University}
\centerline{Mathematics 1160 Applied Calculus}
\centerline{Jim Wolper}
\centerline{6 September 2013}
\centerline{{\bf THREE CUPS OF $E$}}
\centerline{\copyright James S.~Wolper 2012, 2013}

%%%%%%%%%%%%%%%%%%%%%%%%%%%%%%%%%%%%%%%%%%%%%%%

The most important functions in applied mathematics are the
exponential functions.  There are three reasons for this.  First,
the exponential functions directly model phenomena in nature like
radioactive decay, bacterial growth, compound interest, and others.
Less obviously, descriptions of other phenomena like logistic
growth, drug concentrations, and heating involve exponential
functions in an essential way.  

Second (although we won't study this this semester), the
exponential functions serve as a computational model for many of the other
functions that occur in applying mathematics.

Third, exponential functions come up in a natural way when
we do calculus.  Deeper investigation into functions
often leads to surprising relationships.  Is it obvious that
$\displaystyle {1\over{x^2 + 1}}$ is related to trigonometry?
But it is.  Is it obvious that $\displaystyle 1\over x$ is 
related to exponentials?  {\sl But it is!\/}

While models are incredibly useful, modelling is not measurement
In fact, neither is possible without the
other.  But the exponential models are continuous, but we measure
at intervals, not continuously.  Either way, the function involved is
an exponential function, and we need to be 
able to compare the continuously--measured
model to the discretely--measured data, requiring the algebra of
exponential functions, as well as logarithms.

%%%%%%%%%%%%%%%%%%%%%%%%%%%%%%%%%%%%%%%%%%%%%%%
\Sect{The Three Forms}

The exponential functions appear in three major forms.  These
forms are equally useful, but in different circumstances.  They are

\BULL {\bf RELATIVE GROWTH}.  When a quantity grows at a fixed
rate each time period ({\it e.g.\/}, mobile phone handset sales are
growing at 6\% per year), the exponential function implied
has the form

$$H(t) = H_0\cdot (1 + a)^{kt},$$
where $a$ is the growth rate (which may be negative) and $k$
is a constant.

This form is typically found in measurement.

%%%%%%%%%%%%%%%%%%%%%%%%%

\BULL {\bf HALF-LIFE and DOUBLING}

If the amount $Q$ has a half-life $h$, then 

$$Q(t) = Q_0 \cdot \biggl ( {1\over 2} \biggr )^{t/h},$$
while if the quantity $P$ has a doubling time of $H$

$$P(t) = P_0 \cdot 2^{t/H}.$$

This is a convenient form for working with decaying substances.
For example, how long should a pilot wait after taking a drug that might interfere
with flying before flying again?  The standard answer is 
``three half-lifes."  (These are metabolic half-lifes, not
physical half-lifes.)

% source Bruce Chien, personal communication


%%%%%%%%%%%%%%%%%%%%%%%%%%%%%%
\BULL {\bf CALCULUS}

The best exponential function for doing Calculus is

$$f(x) = f_0\cdot e^{rx}.$$

%%%%%%%%%%%%%%%%%%%%%%%%%%%%%%%%%%%%%%%%%%%%%%%
\Sect{Manipulations}

All three forms of the function are useful, and it is not unusual
to need all three in one problem.  So, it is important to 
move between the forms fluently.  There are two key facts that
make this possible:

$$\matrix{
e^{A + B} & = & e^A e^B\cr
e^{\ln x} & = & x\cr
}$$

Remember, the last equation does not say that $e$ and $\ln$ cancel;
it says that the composition of these two functions has the
desired result.


%%%%%%%%%%%%%%%%%%%%%%%%%%%%%%%%%%%%%%%%%%%%%%%
\Sect{Example}

Cardiolite (technetium Tc 99m MIBI) is
a radionuclide used in cardiac imaging.  
According to {\tt http://www.pharmgkb.org}, cardiolite's
effective half-life in the liver is 28 minutes, that is,
after 28 minutes, half of the cariolite in the liver has been
eliminated.  (This measurement combines metabolic and
radioactive decay).  We will describe the 
decay of cardiolite all three ways.

First, define $C(t)$ to be the amount of cardiolite in the 
liver $t$ minutes after injection, and let $C_0$ be
$C(0)$, as usual.  From here, it is easy to write down
the {\bf half life} form of an expression for $C$, namely

$$C(t) = C_0 \cdot \biggl ( {1\over 2} \biggr ) ^{t/28}.$$
You can verify this by noting that $C(28) = C_0 \cdot (1/2)$,
that is, half of the cardiolite is gone.

For the {\bf calculus} form, write $1/2$ as $e^{\ln 1/2}$:

$$\matrix{
  C(t)  & = &  C_0 \cdot \bigl ( {1\over 2} \bigr ) ^{t/28}& ~\cr
        & = & C_0 
               \bigl (
                 e^{\ln 1/2}
               \bigr )
               ^{t/28}\cr
        & = & C_0 e^{{{t}\over{28}}\ln 1/2}\cr
}    % matrix
$$

The constant $\displaystyle {{\ln 1/2}\over{28}}$ is approximately 
$-0.0248$, so

$$C(t) \approx C_0 e^{-0.025t}.$$

%%%%%%%%%%%%%%%%%%%%%%%%%%%%%%
\Sect{Measurement and Modelling}

Researchers draw blood at
intervals and measure the amount of a substance undergoing
decay like cardiolite.
Also, while many population models are in the form $P_0e^{kt}$, more
typically you measure the population in intervals (every minute, every year, $\ldots$).
So you need to go from relative growth rate (what you {\sl
measure\/}) to continuous growth rate (what you {\sl model\/})
and {\it vice--versa\/}.
For typical growth rates, these are very close, but not exactly the same.


For the minute-by-minute decay of cardiolite, write
$C(t) = C_0\cdot(1 + a)^t$ and solve

$$\matrix{
 (1 + a)^t    & = & e^{-0.025t}& {\rm\ take~logs}\cr
 t\ln (1 + a) & = & -0.025t\cr
 \ln(1 + a)   & = & -0.025\cr
 1 + a        & = & e^{-0.025}\cr
 a            & = & e^{-0.025} - 1 \cr
              & \approx & -0.0247\cr
}   % matrix
$$
The minute-by-minute decline of cardiolite in the liver is 
around $2.47\%$.

Notice that the minute-by-minute decay rate is $-0.0247$
while the continuous decay rate is $-0.0248$, which
is very close, but different.  In general, if $x$ is small,
$\ln(1+x) \approx x$, so that is the expected behavior

Finally, how do we get from measurement to half-life?
Given the minute-by-minute decay rate of $-0.0247$,
%
$$\matrix{
(1 - 0.0247)^t & = & \left ( {1\over 2}\right)^{t/h}&{\rm\ [where\ }h{\rm\ is\ unknown]}\cr
t\ln(0.9753) & = & {t\over h} \ln(1/2)\cr
h & = & {{ln(1/2)}\over{\ln(0.9753)}} & ]t{\rm\ cancelled}]\cr
   & \approx & 28,
}$$
which is where we began.

%%%%%%%%%%%%%%%%%%%%%%%%%%%%%%%%%%%%%%%%%%%%%%%
\bye
