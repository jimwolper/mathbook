\magnification 1200
\input /Users/jimwolper/TeX/standard.tex

\font\bigger=cmr10 scaled\magstep1
\font\smaller=cmr7
\def\Tau{\rm T}
\input epsf


%%%%%%%%%%%%%%%%%%%%%%%%%%%%%%%%%%%%%%%%%%%%%%%

\centerline{Idaho State University}
\centerline{Mathematics 1160 Applied Calculus}
\centerline{Jim Wolper}
\centerline{26 September 2012}
\centerline{{\bf Working with Exponentials}}
\centerline{\copyright James S.~Wolper 2012}

%%%%%%%%%%%%%%%%%%%%%%%%%%%%%%%%%%%%%%%%%%%%%%%

\Sect{}

I told my class that the metabolic half--life of nicotine is 2 hours, 
so they know that if the initial amount is $Q_0$ then the amount
$Q(t)$ of nicotine after $t$ hours is
%
$$
Q_0 \left ( {1\over 2} \right )^{t/2}.
$$

But now there's a problem in the book saying that
$Q(20) = 0.36$ while $Q'(20) = -0.002$, where the
time is in minutes.

Which one of us is lying?

First, change the time unit to minutes; change the name of the function to  $f$
while you're at it.  So,
%
$$
f(t) = Q_0 \left ( {1\over 2} \right )^{t/120},
$$
and the book says $f(20) = 0.36$ and $f'(20) = -0.002$

\subSect{Crunching the data}

It's probably easier to think about $f'$ by rewriting $f(t)$
as $Q_0e^{-t\ln(2)/120} \approx Q_0 e^{-0.005776t}$.
Then
%
$$
f'(t) = \left ( {{-t\ln 2}\over{120}}\right ) Q_0 e^{-t\ln(2)/120},
$$
which is just $(-t\ln(2)/120) f(t)$, as one would expect for a
decay problem in which the rate is proportional to the amount 
remaining.  In other words, the model predicts $k = -0.005776$.

\bigskip

Looking at $f$ and $f'$ side-by-side saves a bunch of work.  In general
%
$$
\Biggl \{
\matrix{
0.36 & = & Q_0 e^{kt} \cr
-0.002 & = & k Q_0 e^{kt} \cr
}
$$
so the data says that $k$ is $-0.002/0.36 \approx -0.005556.$

But the 120 minute half life predicted a $k \approx0.005776$; the data
gives a $k$ less than $4\%$ different, which is pretty good.

What is the half-life predicted by the data?  This involves solving
$-\ln(2)/H = -0.5556$ for $H$; the solution is 125 minutes, which is little
over 4\% more than I told them.

%%%%%%%%%%%%%%%
\subSect{Another Kind of Error}

So is there any error at all?  I told the class that the half-life was 2 hours,
freely confessing that my source was {\tt wikipedia.org}.  {\sl That's only one significant figure!\/}
Put differently, converting the 125 minute half--life to hours yields
a half-life of {\sl 2 hours\/}, exactly what I told them.


Going back to the estimation of $k$ from the data but carrying fewer significant digits
leads to a prediction of $0.0056$ and an estimate of $0.0056$, 
which is exact agreement.



%%%%%%%%%%%%%%%%%%%%%%%%%%%%%%%%%%%%%%%%%%%%%%%
\bye
